\chapter{CDDL Specification of the Protocol Messages}
\label{CBOR-section}
\hsref{ouroboros-network/test/messages.cddl}
\label{included-cddl}
This Sections contains the CDDL~\cite{cddl} specification
of the binary serialisation format of the network protocol messages.

To keep this Section in close sync with the actual Haskell implementation
the names of the Haskell identifiers have been reused for the corresponding
CBOR types (with the first letter converted to lower case).
Note, that, for readability, the previous Sections used simplified message identifiers,
for example {\tt RequestNext} instead of {\tt msgRequestNext}, etc.
Both identifiers refer to the same message format.

All transmitted messages satisfy the shown CDDL specification.
However, CDDL, by design, also permits variants in the encoding that are not valid in the protocol.
In particular, the notation ${\tt [} ... {\tt ]}$ in CDDL can be used for both fixed-length
and variable-length CBOR-list, while only one of the two encodings is valid in the protocol.
We add comments in specification to make clear which encoding must be used.

Note that, in the case of the request-response mini protocol (Section~ref{request-response-protocol})
there is only one possible kind of message in each state.  This means that
there is no need to tag messages at all and the protocol can directly transmit
plain request and response data.

\lstinputlisting{../../ouroboros-network/test/messages.cddl}
