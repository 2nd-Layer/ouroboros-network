\documentclass{report}
\usepackage[margin=2.5cm]{geometry}
\usepackage{amsmath, amssymb, stmaryrd, latexsym, amsthm, mathtools}
\usepackage{mathpazo, times}
\usepackage{float}
\usepackage{listings}
\usepackage{url}
\usepackage{natbib}
% \usepackage{parskip} % very ugly with lemmas, invariants, etc without intervening text
\usepackage[disable]{todonotes}
\usepackage{slashed}
\usepackage{tikz}
\usetikzlibrary{automata, positioning, arrows}
\tikzset{
    state/.style={
           rectangle,
           rounded corners,
           draw=black, very thick,
           minimum height=2em,
           inner sep=2pt,
           text centered,
           },
}

\usepackage{forest}
\usepackage{IEEEtrantools}
\usepackage{microtype}
\usepackage{graphicx,color}

\usepackage{hyperref}
\hypersetup{
  colorlinks=false,
  linkcolor={blue},
  citecolor={blue},
  urlcolor={blue},
  linkbordercolor={white},
  citebordercolor={white},
  urlbordercolor={white}
}
\usepackage[capitalise,noabbrev,nameinlink]{cleveref}

% https://tex.stackexchange.com/questions/132823/ieeetrantools-clash-with-cleveref
\makeatletter
\let\if@IEEEissubequation\iffalse
\makeatother

\usetikzlibrary{arrows}

\newcommand{\coot}[1]{\textcolor{violet}{\emph{#1}}}
\newcommand{\njd}[1]{\textcolor{purple}{\emph{#1}}}
\newcommand{\avieth}[1]{\textcolor{blue}{\emph{#1}}}
\newcommand{\dcoutts}[1]{\textcolor{orange}{\emph{#1}}}
\addtolength{\marginparwidth}{-0.1\marginparwidth}

\newcommand{\var}[1]{\mathit{#1}}
\newcommand{\type}[1]{\mathsf{#1}}
\newcommand{\powerset}[1]{\mathbb{P}(#1)}
\newcommand{\order}[1]{\mathcal{O}\left(#1\right)}
\newcommand{\restrictdom}{\lhd}
\newcommand{\subtractdom}{\mathbin{\slashed{\restrictdom}}}
\newcommand{\restrictrange}{\rhd}

\DeclareMathOperator{\dom}{dom}
\DeclareMathOperator{\range}{range}
\DeclareMathOperator*{\argmin}{arg\,min} % thin space, limits underneath in displays
\DeclareMathOperator*{\minimum}{min}
\DeclareMathOperator*{\maximum}{max}

% Number within sections, and don't have separate counters for separate environments
\theoremstyle{definition}{
  \newtheorem{lemma}{Lemma}[section] % Number within sections
  \newtheorem{definition}[lemma]{Definition}
}
\theoremstyle{theorem}{
  \newtheorem{invariant}[lemma]{Invariant}
  \newtheorem{proofobligation}[lemma]{Proof Obligation}
}

\Crefname{invariant}{Invariant}{Invariants}

\numberwithin{equation}{lemma}

%\floatstyle{boxed}
%\restylefloat{figure}

\lstset{basicstyle=\ttfamily\small}

\raggedbottom

\begin{document}

\title{The Shelley Network Layer\\
       {\small (Version 0.1)} \\
       {\large \sc An IOHK technical report}}
\author{Duncan Coutts \\ {\small \texttt{duncan@well-typed.com}} \\
                         {\small \texttt{duncan.coutts@iohk.io}}
   \and Alex Vieth \\ {\small \texttt{alex@well-typed.com}}
   \and Neil Davies \\ {\small \texttt{neil.davies@pnsol.com}} \\
                       {\small \texttt{neil.davies@iohk.io}}
   \and Marcin Szamotulski \\ {\small \texttt{marcin.szamotulski@iohk.io}}
   \and Karl Knutsson \\ {\small \texttt{karl.knutsson@iohk.io}}
   \and Marc Fontaine \\ {\small \texttt{marc.fontaine@iohk.io}}
   }
\date{December 20, 2018}

\maketitle

\begin{abstract}
  This document describes the Shelley network protocol.
\end{abstract}

\tableofcontents

\section*{Version history}

\begin{description}
\item[Version 0.1, Dez 20, 2018  Draft of the table of contents.]
                                  
\end{description}

\chapter{Overview}
\section{Layout of the Document}
\begin{itemize}
\item What goes in which section ?
\item In which order to read ?
\item Which sections can be skipped ?
\end{itemize}
\section{Notation}

\chapter{Requirements}
\subsection{Classes of Participants}
\subsection{Stake pool} % lookup what this is called in the protocols.tex
\subsection{Small stakeholder}
\subsection{User who has delegated}

\section{Node to Node and Node to Consumer}
\begin{itemize}
\item Different protocols for different kinds of participant
 
\item Design discussions in the discussions section.
\end{itemize}

\section{Nodes}
\section{Ouroboros}
\section{Delegation}

\chapter{System Architecture}
\section{Overview}
\section{Design Choices}
\begin{itemize}
\item Only the design choices that have been taken.
\item Design discussions in the discussions section.
\end{itemize}
\section{Nodes}
\section{Protocol Layers}
\section{Components}
\section{Interfaces Between the Layers}

\chapter{Infrastructure}
Specific assumptions about the infrastructure that are relevant for the discussion.
\section{Internet}
\section{Firewall}
\section{TCP}
\section{Operating Systems}

\chapter{Protocols}
\section{State-machine Framework}
\section{Chain Following Protocol}
\section{Block Retrieval Protocol}
\section{What other Protocols ?}
\section{Peer Discovery}
\section{Binary Formats and Low-Level Protocols}
The binary formats are automatically derived from the Haskell data types
That are used.

\chapter{Haskell}
The network protocol itself does not rely on Haskell
But, at the same time, the protocol is being developed in parallel with the
Haskell reference implementation.


\chapter{Discussion}
Alternative view: Exploratory work.
The real work goes here
The Why is at least as important as the What.
\section{Overview}
\section{Design Discussion}
\section{Requirements}
\section{Thread Vectors}
\subsubsection{Asymptotic Resource consumption}
\subsection{Results from Simulations}
\section{Pub Sub}
\section{Of the Shelf Protocols}
\section{Meta Requirements}
\subparagraph{Work in Progress}
This document is evolved in parallel with the work on the protocol design and
the reference implementation.

\subparagraph{The Document should be Comprehensive}
\begin{itemize}
\item Top down approach.
\item Provide the big picture.
\item Usable as a reference point for a broader discussion.
\item Cover every aspect that is related to network connections.
\item Every aspect should at least have a place in the table of contents.
  If there are holes and parts that are not covered the document should say what is missing.
\item Stand alone readable with links to where missing pieces can be found.
\end{itemize}

\subparagraph{Detailed}
\begin{itemize}
\item Sufficient details to allow for new independent implementations that are compatible with
the reference implementation
\item Language agnostic (it is save to skip the Haskell specific parts)
\item Design discussions
\end{itemize}
\subparagraph{Structured}
\begin{itemize}
\item Parts of the document should be in a logical connection
\end{itemize}
\subparagraph{Workflow}

\bibliographystyle{apalike}
\bibliography{references}

\appendix

\end{document}
